\documentclass{article}
\usepackage[landscape,centering]{geometry}

\geometry{
    paperwidth=3in,
    paperheight=5in,
    margin=0.5cm
}

\begin{document}\sloppy
\section{Title}
\begin{itemize}
    \item Introduction. Medici Ventures, blockchain.
    \item Explain how this talk came to be. Differing opinions. Poorly written
        and organized tests.
    \item Developers have love/hate relationship with tests.
    \item Focus is on unit testing.
\end{itemize}

\newpage
\section{Opinionated}
\begin{itemize}
    \item Intended as humour.
    \item Disagreements are opportunities for learning.
\end{itemize}

\newpage
\section{Screed}
\begin{itemize}
    \item Screed is a cool word.
    \item Your mileage may vary.
\end{itemize}

\newpage
\section{I Hate UT}
\begin{itemize}
    \item Developers seem to hate unit tests.
    \item Often an afterthought. Don't leave time for writing tests.
    \item Unit tests are often stale.
\end{itemize}

\newpage
\section{UT Are Not IT}
\begin{itemize}
    \item Wife asked ``what does a unit test do...''
    \item Two distinct kinds of tests: Unit tests and integration tests.
    \item Unit tests test our own units of code, hopefully isolated.
    \item Integration tests, also often called ``system'' or ``functional''
        tests, are full stack tests winding it's way through all of our code
        paths.
    \item Integration tests are often fragile.
\end{itemize}

\newpage
\section{Not UT If...}
\begin{itemize}
    \item Test is not a unit test if...
    \item It talks to a database.
    \item It communicates across the network.
    \item It touches the filesystem.
    \item It can't run at the same time as your other unit tests.
    \item You must do special things to your environment.
    \item Michael Feathers.
\end{itemize}

\newpage
\section{Unit Tests Should Be FAST!}
\begin{itemize}
    \item Waiting 30 minutes for tests to run is 30 minutes of your life you'll
        never get back.
    \item It means you will avoid running tests.
\end{itemize}

\newpage
\section{Unit Tests Should Be Segregated}
\begin{itemize}
    \item Unit tests and integration tests should be segregated.
    \item They serve different purposes. Mingling them makes it impossible to
        run them separately.
    \item Integration tests require external dependencies that are difficult to
        implement in a developer environment.
    \item During development, devs should be running unit tests often. Segregation
        of unit and integration tests reduces friction.
\end{itemize}

\newpage
\section{Unit Tests Should Assert One Thing}
\begin{itemize}
    \item Unit tests should be simple.
    \item Multiple assertions make it difficult to understand what failed.
    \item The exception is when multiple assertions are used to confirm results
        that are related.
\end{itemize}

\newpage
\section{Unit Tests Should Be Run Often}
\begin{itemize}
    \item Before modifying code.
    \item After code change.
    \item At night before you leave.
    \item In the morning when you arrive.
    \item Just because...
\end{itemize}

\newpage
\section{Tips}
\begin{itemize}
    \item Adopting some best practices can make unit tests more useful.
\end{itemize}

\newpage
\section{Tips: Naming}
\begin{itemize}
    \item Two hardest things in computer science...
    \item Unit test names should act as a specification.
    \item Describe what is expected.
    \item Helps locate the problem on failed test.
\end{itemize}

\newpage
\section{Tips: Comments}
\begin{itemize}
    \item Commenting your tests improves readability.
    \item Helps document your units.
\end{itemize}

\newpage
\section{Tips: Isolate}
\begin{itemize}
    \item Mocks, stubs, fakes, test doubles.
    \item Loose coupling helps with this.
\end{itemize}

\newpage
\section{Tips: Overspecification}
\begin{itemize}
    \item Leads to tests coupling strongly to production code and limiting
        ability to refactor.
    \item Sign of this is testing implementation details instead of the overall
        behavior.
    \item Overusing mocks is an indicator of overspecification.
    \item Prefer stubs over mocks as much as possible.
    \item Test should focus on testing the results of a function instead of the
        internal actions. Not always possible. Refer to design.
\end{itemize}

\newpage
\section{Tips: Brevity}
\begin{itemize}
    \item Tests should be short and to the point.
    \item Long elaborate tests indicate overspecification or the code is in
        need of refactoring.
    \item Brevity helps us maintain focus on tests as documentation.
\end{itemize}

\newpage
\section{Tips: Setup}
\begin{itemize}
    \item Setup common dependencies for each suite of tests.
    \item Teardown resets dependencies.
    \item Prevents fragile unit tests by breaking dependencies between tests.
\end{itemize}

\newpage
\section{Tips: Arrange Act Assert}
\begin{itemize}
    \item Perform any test specific setup that needs to be done.
    \item Call the unit.
    \item Assert to confirm the single result, even if multiple asserts are
        needed for this purpose.
\end{itemize}

\newpage
\section{Tips: DAMP vs DRY}
\begin{itemize}
    \item Don't Repeat Yourself.
    \item Declarative And Meaningful Phrases
    \item Each test is an isolated chapter in your test book. If removing a
        piece of logic hurts the readability, then don't remove it.
    \item From a maintenance perspective, don't centralize a piece of logic that
        can't be changed on it's own.
\end{itemize}

\newpage
\section{Tips: Perfection}
\begin{itemize}
    \item Remember, you're trying to run a business!
    \item Perfection takes a lot of time. Don't waste time.
    \item The test can be refactored if it fails when it shouldn't.
    \item Insufficient isolation can be fixed later if necessary.
\end{itemize}

\newpage
\section{Tips: Code Coverage}
\begin{itemize}
    \item Some devs don't like code coverage.
    \item I get a kick out of seeing my reports improve.
    \item Ideas for tests often come from the coverage report.
    \item How do you know what you're really testing?
    \item Code coverage reports help reason about design and guide refactor.
    \item Spot trends in testing coverage over time.
    \item Test quality != code coverage.
\end{itemize}

\newpage
\section{Tips: Test Driven Development}
\begin{itemize}
    \item Does not mean Test Deficit Disorder.
    \item Test last: Risks, design problems and bugs discovered late. Can
        descend into frustrating spiral of debugging and rework.
    \item Test last leads to heavy refactoring burdens.
    \item You might not leave time for adequate test design, leading to being
        rushed. May miss important cases.
    \item Test first? I haven't even figured out what to name stuff yet...
    \item Test first forces you to specify what you are building in advance
        of building it.
    \item Risk of rework. Mini waterfall. Discoveries during development that
        make you want to iterate.
    \item Test driven development. Designing tests and code together.
    \item Allow the design of the tests and the code emerge as you develop.
    \item TDD, ahhh, nirvana. Iterative cycle. Helps guide to clear design.
    \item Poorly designed code is hard to test.
    \item If writing unit tests is hard, your code is probably poorly designed.
\end{itemize}

\newpage
\section{Tips: Limitations}
\begin{itemize}
    \item Unit tests will not prove correctness. GIGO.
    \item Unit tests won't find integration errors.
    \item Unit tests will not test non-functional requirements like performance
        and security.
\end{itemize}

\newpage
\section{Purpose: Why Do We Do It?}
\begin{itemize}
    \item Your own personal development process.
    \item Developer responsibility. Not optional extra or job of seperate test team.
    \item Avoid embarrassing bugs in your code that others find.
    \item To help your team not break your code
    \item Writing code is how you tell your colleagues how you feel about them.
    \item Psychopathic developer who knows where you live.
\end{itemize}

\newpage
\section{Purpose: What To Build}
\begin{itemize}
    \item Unit tests cannot be written without a good understanding of what to
        build. Forces business requirements.
\end{itemize}

\newpage
\section{Purpose: Better Design}
\begin{itemize}
    \item Anecdote about PDF Form Filler being untestable.
    \item Well designed code is testable. If your code is hard to test, it is
        not well designed.
    \item Decompose the problem into units that are independently testable.
        (loose coupling)
    \item Forces thinking of interface seperately from implementation.
    \item Just the discipline of writing unit tests helps with better design.
\end{itemize}

\newpage
\section{Purpose: Documentation}
\begin{itemize}
    \item Unit tests document our units. Comments help with this.
    \item Unit tests should be highly readable.
    \item Think of them as an executable specification. Tests document the
        behavior of the code in a form you can execute.
    \item Demonstrates how a unit is intended to be used, how to construct
        it, call the method, what kind of arguments, what kind of results.
\end{itemize}

\newpage
\section{Purpose: Fear-Free Refactor}
\begin{itemize}
    \item Adding new functionality or just refactoring units is easier. Unit
        tests let us know quickly when something is broken.
\end{itemize}

\newpage
\section{Purpose: Regression Protection}
\begin{itemize}
    \item Something that worked before and now doesn't.
    \item Changes to unknown dependencies can cause failures.
    \item Test failure messages should be very clear and readable so it is
        easy to find any failures.
\end{itemize}

\newpage
\section{Purpose: I Love Unit Tests}
\begin{itemize}
    \item Unit tests are your friend. They make your life easier.
    \item Unit tests are challenging and fun!
    \item Unit tests make you a better developer.
    \item Unit tests earn the admiration of your colleagues.
    \item Unit tests are 100\% organic, non-GMO, and gluten-free.
    \item Unit tests are both good and good for you.
\end{itemize}
\end{document}
